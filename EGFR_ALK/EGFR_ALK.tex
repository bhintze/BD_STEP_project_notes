\documentclass[12pt]{article}

\bibliographystyle{unsrt}

\title{Notes on EGFR and ALK}
\author{Bradley Hintze}
\date{\today}

\begin{document}
\maketitle

\section{Epidermal Growth Factor Receptor (EGFR)}
\subsection*{Overview:}
Epidermal Growth Factor Receptor (EGFR) is a cell surface receptor that is activated when it binds its ligand causing homodimerization \cite{Yarden1987}. This dimerization causes activation if the inner-cellular C-terminal kinase activity causing autophosphorylation of C-terminal tyrosines Y992, Y1045, Y1068, Y1148 and Y1173 \cite{Downward1984}. This phosphorylation acts as a siglnal to mainy pathways involved in cell migration, adhesion, and proliferation \cite{Oda2005}.

\subsection*{Implications in NSCLC:}
Mutations in EGFR have been linked to squamous-cell carcinoma \cite{Walker2009}. The drug gefitinib has been shown to be a stong inhibitor of L858R mutant of EGFR \cite{Paez2004}. Bradley needs to do more research!

\noindent
\subsection*{Questions Pending}
\begin{itemize}
\item Read this \cite{Rosell2009}.
\item Are there published guidlines for using gefitinib or erlotinib?
\item What mutations are (and are not) sensative to gefitinib and erlotinib?
\item Paez et al. showed that mutations in EGFR to be more common in japanese (asians?) compared to european decendants and women compared to men \cite{Paez2004}. Can we show frequency of mutation vs. race/sex?
\item Is targeted treatment bing misused, i.e. is gefitinib being used on patients lacking the EGFR  L858R mutant? If so, are survival rates as expected -- lower than the group having the L858R mutant?
\end{itemize}

\section{Anaplastic Lymphoma Kinase (ALK)}
Bradley needs to do research!

\newpage
\bibliography{EGFR_ALK}
\end{document}